%\documentclass[10pt,letterpaper,twoside,openright]{book}
%\usepackage[spanish]{babel}
%\usepackage{amsmath}
%\usepackage{amsfonts}
%\usepackage{amssymb}
%\usepackage{nomencl}
%\usepackage{geometry} 
%\usepackage[latin1]{inputenc} 
%\usepackage{verbatim}
%\usepackage{amsthm}
%\usepackage{graphicx}
%\usepackage{datetime}

%\begin{document}
	
\begin{titlepage}

	\begin{center}

		\huge{\color{red}UNIVERSIDAD DE GUANAJUATO}
		\vspace*{0.5cm}
	
		\huge{\color{red}CAMPUS GUANAJUATO}
		\vspace*{0.5cm}		
				
		\huge{\color{red}DIVISI�N DE CIENCIAS NATURALES Y EXACTAS} 	
		\vspace*{0.5cm}	
		
		\includegraphics[scale=1]{escudo.png}

		%\rule{15cm}{0.5mm}			
		%\huge{\emph{Simulaci�n magnetohidrodin�mica de radiogalaxias}}			
		%\huge{Simulaci�n magnetohidrodin�mica de radiogalaxias}			
		%\huge{\emph{INESTABILIDADES GRAVITACIONALES ALREDEDOR DE DISCOS CIRCUNPLANETARIOS}}			
		\huge{\color{red}INESTABILIDADES GRAVITACIONALES ALREDEDOR DE DISCOS CIRCUNPLANETARIOS}			
		%\rule{15cm}{0.5mm}
			
		\vspace*{0.5cm}
		\Large{Tesis presentada al} \\
		\huge{\color{red}DEPARTAMENTO DE ASTRONOM�A}
		\vspace*{0.5cm}		
		
		\vspace*{0.5cm}
		\Large{como requisito para la obtenci�n del grado de} \\
		\huge{\color{red}MAESTR�A EN CIENCIAS (ASTRONOM�A)}
		\vspace*{0.5cm}		

		\vspace*{0.5cm}
		\Large{por} \\
		\huge{\color{red}OSCAR IGNACIO BARRAG�N VILLANUEVA}
		\vspace*{0.5cm}		
		
		\vspace*{0.5cm}
		\Large{asesorado por} \\
		\huge{\color{red}DR. ERICK NAGEL VEGA}
		\vspace*{0.5cm}		

		\vspace*{0.5cm}		
                \huge{Guanajuato, Gto. - September, 2014}		
		\vspace*{0.5cm}		
		
        \vfill		
		
	\end{center}

\end{titlepage}

%\end{document}
