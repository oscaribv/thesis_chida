%Answers to these questions should be found in the abstract: 
%  What did you do? 
%  Why did you do it? What question were you trying to answer? 
%  How did you do it? State methods.
%  What did you learn? State major results. 
%  Why does it matter? Point out at least one significant implication.

\selectlanguage{spanish}%
\begin{abstract}
Se presentan los resultados de simulaciones de chorros relativistas mediante 
un nuevo m�dulo GRMHD incorporado en el c\'odigo \texttt{JIX}.  Observaciones recientes 
de radiogalaxias muestran que los modelos de propagaci\'on de los chorros 
provenientes de AGNs deben incluir din\'amica de fluidos relativistas. 
Usando un formalismo GRMHD, es posible incluir efectos relativistas y gravitacionales en la simulaci\'on num\'erica de los jets. 
Se desarroll\'o un algoritmo para resolver el sistema de ecuaciones GRMHD num\'ericamente
dentro del formalismo 3+1. En este algoritmo se implementa la aproximaci\'on a fluido
perfecto y magnetohidrodin\'amica ideal. La validaci\'on del algoritmo se realiz\'o mediante pruebas 
est\'andar para c\'odigos de este tipo, desde ondas de choque hasta acreci\'on por un
agujero negro 
de Schwarzschild. Los resultados de estas pruebas muestran que el algoritmo
resuelve correctamente el sistema de ecuaciones. 
Se presentan las simulaciones de propagaci\'on de chorros en el medio interestelar de
una galaxia,
obtenido usando este c\'odigo.
La velocidad estimada de propagaci\'on del chorro usando simulaciones es comparable 
con los datos observacionales.  
\end{abstract}

\selectlanguage{english}%
\begin{abstract}
The results of simulations of relativistic jets using a new GRMHD module
incorporated into 
the code \texttt{JIX} are presented. Recent observations of radio galaxies show that, 
relativistic fluid dinamics is required for modelling the propagation of jets in 
Active Galactic Nuclei (AGNs). Using the formalism of GRMHD, it is possible to 
incorporate the relativistic as well as the gravitational effects in the simulations
of jets.
An algorithm was developed to solve the GRMHD equations numerically using the 3+1
formalism. 
This algorithm implements perfect fluid with ideal magnetohydrodinamics.
The validation of the algorithm was achieved using standard tests employed to test
the MHD codes 
such as shock waves, acretion onto a Schwarzschild black hole. 
The results of these tests show that the algorithm solve the system of equations
correctly. 
Simulations of jet propagation in the interestelar medium of a galaxy, 
obtained using this code are presented. The velocity of propagation of the jet
estimated 
using the simulations compares with that obtained from the observations.

\end{abstract}

%\selectlanguage{spanish}%
